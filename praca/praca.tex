\documentclass{SGGW-thesis}
\INZYNIERSKAtrue % set by default
\WZIMtrue % no other flags for departments exist right now

%\usepackage{fontspec}
%\setmainfont{Times New Roman}

\usepackage{hyperref}

\title{Symulacja silnie niestabilnego układu dynamicznego wraz z analizą basenów dopływu}
\Etitle{Simulation of a highly unstable dynamic system with phase space analysis}
\author{Autor Iksiński}
%\thanks{Funded by the Overleaf team.}
\date{Marzec 2024}
%\university{Szkoła Główna Gospodarstwa Wiejskiego\\ w Warszawie}
%\dep{Wydział Zastosowań Informatyki i Matematyki}
\album{208266}
\thesis{Praca inżynierska na kierunku:}
\course{Informatyka}
\promotor{doktora Pawła Hosera}
\pworkplace{Katedra Sztucznej Inteligencji}

\begin{document}
\maketitle
\statementpage
\abstractpage
{Symulacja silnie niestabilnego układu dynamicznego wraz z analizą basenów dopływu}
{Napisano program zderzający elektrony z protonami z różnymi pędami początkowymi i zapisano, co z czym stykło.}
{Symulacja, Układ dynamiczny, Zbiór Fatou, Przestrzeń fazowa, Stabilność strukturalna, Stabilność Lapunowa, Symulacja plazmy, Symulacja elektronów w polu magnetycznym}
{Simulation of a highly unstable dynamic system with phase space analysis}
{A program shooting electron at protons with different momentum has been written and it saves data on which element collides with which.}
{Simulation, Dynamical system, Fatou set, phase space, structural stability, Lyapunov stability, simulating electrons in a magnetic field, plasma simulation}

{
  \doublespacing
  \tableofcontents
}

\startchapterfromoddpage % niezależnie od długości spisu treści pierwszy rozdział zacznie się na nieparzystej stronie

\chapter{Wykaz symboli i skrótów}
TLDR -- zbyt długie, nie czytałem

\chapter{Wstęp}
Coś tam napisane. 

\section{Przegląd literatury}
W \LaTeX-u.\cite{talbot2013}

\chapter{Cel i zakres pracy}
Jeszcze więcej tekstu. Celem pracy jest jej napisanie, rzecz jasna.
\section{Wykorzystane technologie}
C\#, Plotly, Dotnet.ReproducibleBuilds (za: https://www.meziantou.net/creating-reproducible-build-in-dotnet.htm), coś do internacjonalizacji i debugowania interfejsu

\chapter{Założenia metodyczne}
Jeszcze więcej tekstu.

\chapter{Część doświadczalna}
Jeszcze więcej tekstu.

\chapter{Wyniki i dyskusja}
Jeszcze więcej tekstu.

\chapter{Wnioski}
Panie Łukasz, popraw dokumentację tej klasy w LaTeXu.

\chapter{Załączniki}
\section{Załącznik 1}
Bla.

\section{Załącznik 2}
Bla bal bal.

\bibliographystyle{vancouver}
\bibliography{przyklad}
\beforelastpage[2023]
\end{document}
