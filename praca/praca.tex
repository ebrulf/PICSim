\documentclass{SGGW-thesis}
\INZYNIERSKAtrue % set by default
\WZIMtrue % no other flags for departments exist right now

%\usepackage{fontspec}
%\setmainfont{Times New Roman}

\usepackage{hyperref}

\title{Symulacja silnie niestabilnego układu dynamicznego wraz z analizą basenów dopływu}
\Etitle{Simulation of a highly unstable dynamic system with phase space analysis}
\author{Autor Iksiński}
%\thanks{Funded by the Overleaf team.}
\date{Marzec 2025}
%\university{Szkoła Główna Gospodarstwa Wiejskiego\\ w Warszawie}
%\dep{Wydział Zastosowań Informatyki i Matematyki}
\album{208266}
\thesis{Praca inżynierska na kierunku:}
\course{Informatyka}
\promotor{doktora Pawła Hosera}
\pworkplace{Katedra Sztucznej Inteligencji}

\begin{document}
\maketitle
\statementpage
\abstractpage
{Symulacja silnie niestabilnego układu dynamicznego wraz z analizą basenów dopływu}
{Napisano program zderzający elektrony z protonami z różnymi pędami początkowymi i zapisano, co z czym stykło.}
{Symulacja, Układ dynamiczny, Zbiór Fatou, Przestrzeń fazowa, Stabilność strukturalna, Stabilność Lapunowa, Symulacja plazmy, Symulacja elektronów w polu magnetycznym}
{Simulation of a highly unstable dynamic system with phase space analysis}
{A program shooting electron at protons with different momentum has been written and it saves data on which element collides with which.}
{Simulation, Dynamical system, Fatou set, phase space, structural stability, Lyapunov stability, simulating electrons in a magnetic field, plasma simulation}

{
  \doublespacing % a to sprawdzałem, to był standard w Polsce też na długo przed wojną, i w Biblii Gdańskiej po kropce jest dłuższa przerwa
  \tableofcontents
}

\startchapterfromoddpage % niezależnie od długości spisu treści pierwszy rozdział zacznie się na nieparzystej stronie

\chapter{Wykaz symboli i skrótów}
\begin{description}
  \item [TLDR] -- zbyt długie, nie czytałem
  \item [UI] -- interfejs graficzny
  \item [format Word] -- Office Open XML Document
  \item [język RTL] -- język z pismem pisanym domyślnie z prawej na lewą stronę w stosunku do osoby piszącej (inaczej, jak w alfabecie łacińskim)
  \item [GNU] -- projekt wolnej (jak w słowie wolny rynek) implementacji Unixo-podobnego systemu operacyjnego prowadzony przez Free Software Foundation
\end{description}

\chapter{Wstęp}
Zbyt długo się zbieram z napisaniem tego. Do tego stopnia, że przegapiłem pierwszy termin oddania pracy. Jeszcze więcej tekstu do napisania. Celem pracy jest jej napisanie, rzecz jasna.

Jako absolwenta profilu mat-fiz-inf interesuje mnie matematyka, w tym fraktale. Alternatywnym tematem pracy inżynierskiej miał być algorytm do tworzenia wykresów funkcji Bolzano, najstarszej odkrytej funkcji o właściwościach fraktalnych, na zadanym odcinku w euklidesowej przestrzeni dwuwymiarowej. W ten sposób zapełniłbym niszę.

W którym dziele przeczytałem, że silnie niestabilne układy dynamiczne mają poszarpane fraktalnie baseny dopływu? To do ustalenia. 

\section{Przegląd literatury}
W \LaTeX-u.\cite{talbot2013}

\chapter{Cel i zakres pracy}
Program generuje okrężną drogą (symulacji fizycznej) fraktale Newtona. Taki zakres. Celem pracy jest zdobycie tytułu inżyniera i zakończenie przeze mnie edukacji wyższej. Do tego ostatniego celu nie przykładam się, jak powinienem.
\section{Wykorzystane technologie}

\subsection{C\# 12}
Język programowania Microsoftu, z którym miałem najwięcej styczności na studiach. Posługuję się wersją dla \textbf{.NET 8}. Do interfejsu graficznego wykorzystuję \textbf{MAUI}, będąca nowszą biblioteką od Windows Forms. Jako środowisko programistyczne wykorzystałem \textbf{Visual Studio Community 2022}.
\subsection{Plotly}
Biblioteka do tworzenia wykresów i eksportu do pliku graficznego. Zgodna z wybranym przeze mnie językiem.
\subsection{Dotnet.ReproducibleBuilds}
Na ostatnią chwilę dodałem  (za: https://www.meziantou.net/creating-reproducible-build-in-dotnet.htm) zapewnienie zgodności kodu z kompilowanym programem. Najnowszy trend w programowaniu.
\subsection{Microsoft.Extensions.Localization}
Menu jest w języku polskim, jednak chcę umożliwić tłumaczenia. Testy języków RTL (np. arabski, hebrajski) bądź innych, pionowych kombinacji (dawniej chiński i japoński) są nieuwzględnione.
\subsection{XUnit}
Biblioteka do testów jednostkowych, w tym interfejsu graficznego.
\subsection{Git}
System kontroli wersji, w ten sposób zapisuję i cofam postępy w pracy. Repozytorium kodu źródłowego jest też dostępne dla osób postronnych.
\subsection{Github Actions}
Pomyślałem też sobie, że przy każdej wprowadzonej zmianie będę kompilował zarówno program, jak i pracę inżynierską. Jak do tej pory bez powodzenia, ale debugowanie to też nauka.
\subsection{\TeX}
Praca inżynierska jest pisana przy użyciu systemu składu drukarskiego \TeX, zestawu makr \LaTeX, silnika pdfTeX oraz klasy przygotowanej przez pana Łukasza. Z pliku tworzę zarówno plik w formacie PDF, jak i Word. Z racji środowiska pracy korzystam z dystrybucji MiKTeX. Ponadto spis treści tworzę w programie \textbf{MakeIndex}, zaś do zarządzania bibliografią -- \textbf{BibTeX}.
\subsection{SBOM}
Uwzgledniłem też tzw. \textit{Software Bill of Materials}, w celu śledzenia licencji użytych bibliotek, w tym zgodności mojego programu z nimi.
\subsection{Diagramy}
Jakaś forma diagramów, typu UML, została wykorzystana w ramach tworzenia pracy. Posłużyłem się \textbf{Visual Paradigm} bądź podobnym programem.
\subsection{Doxygen + Graphviz}
Programy do automatycznego tworzenia dokumentacji. Pozostałość uniwersyteckiego kursu z inżynierii oprogramowania. W chwili, gdy to piszę, kusi mnie pójść dalej i napisać Unixowy \textit{man page} plus wyposażyć w argumenty konsolowe rodem z GNU typu \texttt{\-\-version} czy \texttt{\-\-help} -- proszę mnie od tego odstraszyć, program nie musi być wspaniały.

\chapter{Założenia metodyczne}
Zadany program ma działać na Windowsie 10, edycji na procesory x64, bez obsługi ekranu dotykowego. Tylko na tą platformę debuguję.

Program składa się z menu z ustawieniami (domyślnie: po lewej stronie) oraz ekranu, na którym wyświetlać ma się symulacja. Ekran domyślnie ładuje się z trzema protonami ustawionymi w trójkąt równoboczny oraz z pojedynczym elektronem. Za pomocą myszki użytkownik nadaje pędy cząsteczkom (przez cząsteczki mam na myśli elektron)(reprezentowane za pomocą strzałki) oraz ich położenie początkowe. W ustawieniach zaś dostosowuje opcje wykresu (format, miejsce zapisu, rozdzielczość), jak również opcja obliczeń wszystkich kombinacji naraz. 

Za pomocą przycisku uruchamiana jest symulacja z zadanymi opcjami. Cząsteczki się poruszają aż do zderzenia bądź ustalonego limitu czasowego. Jest możliwość ponownego odtworzenia symulacji pod warunkiem braku zmian (w pętli?). Po ukończeniu symulacji można podejrzeć na osobnym oknie wynik obliczeń wszystkich kombinacji naraz (jeśli został wybrany), czyli wykres basenów dopływu, docelowo: fraktal Newtona.

Wartość ładunku elektrycznego symulowanych cząsteczek nie ulega zmianie.

\chapter{Część doświadczalna}
To jak to wygląda obsługa tego programu? Jak to było go pisać?

\chapter{Wyniki i dyskusja}
Porównanie z innymi programami i algorytmami. Zalety i wady. Rozmiar, zużycie zasobów.

\chapter{Wnioski}
Dałem radę. Jak po grudzie, ale dałem radę.
Panie Łukasz, popraw dokumentację tej klasy w LaTeXu.

\chapter{Załączniki}
\section{Załącznik 1}
Bla.

\section{Załącznik 2}
Bla bal bal.

\bibliographystyle{vancouver}
\bibliography{przyklad}
\beforelastpage[2025]
\end{document}
